\newpartenx{Answer all questions}

\newpartbm{Jawab semua soalan}

\bigskip 

\question{}
%\partQuestion{}

\renewcommand{\arraystretch}{1.2}	% to make better spacing between rows in the table
\begin{table}[H]
	\caption{\jadual}	% no caption	
	\begin{tabulary}{\textwidth}{L R C}
		\toprule[1.5pt]
		Short sentences & short one  & Long sentences \\ \midrule
		This is short.       & 173 & This is much loooooooonger, because there are many more words.  \\ 
		This is not shorter. & put some word here & This is still loooooooonger, because there are many more words. \\ \bottomrule[1.5pt]
	\end{tabulary} 
	\label{tab:tabletabulary}
\end{table}

Note that the column width is set automatically so that it will wrap long sentences into a few lines as demonstrated in \cref{tab:tabletabulary}. 

\begin{table}[H]\centering
	\caption{\jadual}	% no caption	
	\begin{tabulary}{0.7\textwidth}{|L|R|C|}
		\hline
		\textbf{Short sentences} & \textbf{short one}  & \textbf{Long sentences} \\ \hline
		This is short.       & 173 & This is much loooooooonger, because there are many more words.  \\ \hline
		This is not shorter. & put some word here & This is still loooooooonger, because there are many more words. \\ \hline
	\end{tabulary} 
	\label{tab:tabletborder}
\end{table} 

In \cref{tab:tabletborder}, I make the table to occupy 70\% of paper width. Also, I change the way I draw the border. 

%% Uncomment this table and explanation below the table to see the output. 
\begin{table}[H]\centering 
	\caption{\jadual}	% no caption	
	\begin{tabulary}{0.8\textwidth}{|C|R|C|}
		\hline
		\multicolumn{2}{|c|}{\textbf{Merge 2 columns}} & \textbf{Long sentences} \\ \hline
		\multirow{2}{=}[-4.5mm]{This is short text LoL.} & \multicolumn{1}{c|}{\multirow{2}{*}{173}} & This is much loooooooonger, because there are many more words.  \\ \cline{2-3}
		& put some word here & This is still loooooooonger, because there are many more words. \\ \hline
	\end{tabulary} 
	\label{tab:tablemulticolumnrow}
\end{table} 

\cref{tab:tablemulticolumnrow} set the table width so that it will occupy 80\% of the paper width. Then, I use \verb|multirow| package to span through 2 columns. This package also can be made to adjust the vertical alignment in a table whenever a row occupy more than a single line of sentence. \verb|multicolumn| not only use to merge two or more column, but can also be used to change a properties of a single row and column such as the text alignment and table border. 

\begin{table}[H]\centering
	\caption{\jadual}	% no caption		
	\begin{tabular}{|p{3cm}|r|b{7cm}|}
		\hline
		\multicolumn{2}{|c|}{\textbf{Merge 2 columns}} & \textbf{Long sentences} \\ \hline
		\multirow{2}{*}[-4mm]{This is short.} & \multicolumn{1}{c|}{173} & This is much loooooooonger, because there are many more words.  \\ \cline{2-3}
		& put some word here & This is still loooooooonger, because there are many more words. \\ \hline
	\end{tabular} 
	\label{tab:tablemulticolumn}
\end{table} 

If you want to specify the exact size of each column, then make use of \verb|p{size}|, \verb|m{size}|, or \verb|b{size}|. \texttt{p} means normal cells, they are like parbox with alignment at the top line. \texttt{b} means alignment at the bottom, so the baseline is at the bottom line. \texttt{m} means alignment in the vertical center, i.e. the baseline is in the center. However, they not work very well with \verb|multitrow|. An example is given in \cref{tab:tablemulticolumn}.

\paperend	%print -oooOooo- properly